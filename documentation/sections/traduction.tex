\documentclass[../informe.tex]{subfiles}
\begin{document}
Per traduir el problema a WPMS es crearà una instància de WCNFFormula(). Així doncs, es traduirà la lectura que s'ha realitzat del fitxer. \\

La traducció es divideix en dos blocs:
\begin{itemize}
  \item \textit{Soft Clauses.}
  \item \textit{Hard Clauses.}
\end{itemize}

\subsubsection{Soft Clauses}
\label{sections:soft_clauses}
S'iterarà la llista creada al llegir el fitxer que conté els noms dels paquets. Per cada iteració es realitzaràn dues accions:
\begin{itemize}
  \item Guardar el nom del paquet en un diccionari, assignant-li com a valor una nova variable de la fórmula de WPMS.
  \item Afegir a la frórmula WPMS una clàusula que contingui com a literal la variable creada en l'acció anterior, amb pes 1.
\end{itemize}

\subsubsection{Hard Clauses}
S'hauràn de realitzar dues iteracions, una a la llista de les dependències i una a la llista dels conflictes. \\
\begin{itemize}
  \item \textbf{Llista de les dependències:} Per cada iteració s'obtindrà una llista amb una dependència. Es traduiràn els noms dels paquets de la llista a variables de la fórmula mitjançant el diccionari obtingut a l'apartat \ref{sections:soft_clauses}. Així doncs, tindrem una clàusula que se li ha de negar el primer literal. Un cop negat ja es podrà afegir a la fórmula amb pes infinit. \\
  \item \textbf{Llista de les dependències:} Per cada iteració s'obtindrà una llista amb un conflicte. Es traduiran els nomes dels paquets de la llista a variables de la fórmula mitjançant el diccionari obingut a l'apartat \ref{sections:soft_clauses}. Cada traducció dels noms dels paquets s'haurà de negar. Per tant, s'obtindrà una clàusula amb tots els literals negats, que s'haurà d'afegir a la fórmula amb pes infinit.
\end{itemize}

\end{document}
