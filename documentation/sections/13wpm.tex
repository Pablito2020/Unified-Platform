\documentclass[../informe.tex]{subfiles}
\begin{document}

La implementació del mètode \textit{to\_13wpm$()$} consta de dos blocs principals, tot i que comparteixen part de la implementació:

\begin{itemize}
  \item \textit{Hard clauses}
  \item \textit{Soft clauses}
\end{itemize}


% HARD CLAUSES SECTION %

\subsection{Hard Clauses}
\label{subsection:hard_clauses}

Primer ens centrarem en el primer bloc, el de les \textit{hard clauses}. Al passar una \textit{hard clause} a notació \textit{1,3} s'hauràn de tenir en compte tres casos:
\begin{itemize}
  \item El nombre de literals és inferior a 3.
  \item El nombre de literals és igual a 3.
  \item El nombre de literals és superior a 3.
\end{itemize}

En el primer cas s'haurà de crear una nova clàusula duplicant o triplicant, segons convingui, un dels literals, fent així que aquesta consti exactament de tres literals. Per suposat, al ser una \textit{hard clause} el pes serà infinit.\\

En el segon cas, com que el nombre de literals és tres, no serà necessari fer cap modificació. \\

En el tercer i últim cas es seguirà el següent procediment:

\begin{enumerate}
  \item Crear una clàusula amb tres literals: dos de la clàusula a transformar i un d'auxiliar.
  \item Comprovar quants literals queden a la clàusula inicial:
        \begin{enumerate}
          \item \textbf{= 2: } Afegir el literal auxiliar de la clàusula anterior negat i els dos literals que resten de la clàusula inicial.
          \item \textbf{> 2: } Afegir el literal auxiliar de la clàusula anterior negat, un dels literals de la clàusula inicial i un nou literal auxiliar.
        \end{enumerate}
  \item Realitzar el pas 2 fins que no restin literals a la clàusula inicial.
\end{enumerate}


% SOFT CLAUSES SECTION %

\subsection{Soft Clauses}
En primer lloc es crearà una clàusula amb una variable de reificació que substitueixi als literals de la clàusula a transformar. L'únic literal de la nova clàusula haurà de ser negat, i tindrà el mateix pes que el de la clàusula transformada.\\

En segon lloc es crearà una nova clàusula amb els literals substituits anteriorment i la variable de reificació (en aquest cas sense negar), amb pes infinit. Així doncs, s'obtindrà una \textit{hard clause} la qual tractarem igual que s'ha descrit en l'apartat \ref{subsection:hard_clauses}.

\end{document}
