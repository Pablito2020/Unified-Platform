\documentclass[../informe.tex]{subfiles}
\begin{document}
Primerament es llegirà el fitxer d'entrada, el qual constarà de cinc tipus de línies diferents segons el primer caràcter de la línia:

\begin{itemize}
  \item \textbf{\#:} Línia de comentari. No es realitzarà cap acció sobre aquesta línia.
  \item \textbf{p:} Línia en que s'especificarà el nombre de paquets. Es guardarà l'última paraula, la qual serà el número de paquets que s'especificaràn a continuació.
  \item \textbf{n:} Línia en que s'especificarà el nom del paquet. Es guardarà el nom del paquet a una llista.
  \item \textbf{d:} Línia en que s'especifiquen les dependències. Es guardarà la dependència com una llista dins una llista.
  \item \textbf{c:} Línia en que s'especifiquen els conflictes. Es guardarà el conflicte com una llista dins una llista.
\end{itemize}

\end{document}
