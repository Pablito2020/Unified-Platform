\documentclass[12pt, letterpaper]{article}
\usepackage{graphicx} % needed for images
\usepackage[export]{adjustbox} % needed for adjustable images
\usepackage{flafter} % needed for figures
\usepackage{hyperref} % needed for a clickable table of contents
% \usepackage{cleveref}
\usepackage{polyglossia} % needed for catalan support
\usepackage{subfiles} % needed for a subfile structure
\usepackage[table]{xcolor}
\usepackage{fancyhdr} % needed for fancy header 
\usepackage{listings}
\usepackage{tabularx}
\usepackage{array}
\usepackage{color}
\usepackage{colortbl}
\usepackage{lineno}
\usepackage{subfig}

\setmainlanguage{catalan}
\graphicspath{ {images/} }

\pagestyle{fancy}
\fancyhf{}
\fancyhead[LE,RO]{Enginyeria del Programari}
\fancyhead[RE,LO]{Pràctica 4: Proves Unitàries}
\fancyfoot[LE,RO]{\thepage}
\renewcommand{\headrulewidth}{1pt}
\renewcommand{\footrulewidth}{1pt}

% information
\title{%
    \begin{center}
	\includegraphics[width=4cm,height=3cm]{udl.png}
    \end{center}
    \line(1,0){250}\\[0.3cm]
    \textbf{Pràctica 4: Proves Unitàries}
    \line(1,0){250}
    \\[0.5cm]
	\large Enginyeria del Programari - Grau en Enginyeria Informàtica
}
\author{Roger Fontova Torres, Pablo Fraile Alonso i Oscar Salcedo Heredia}
\date{\today}

% document
\begin{document}
    
% title
\maketitle
\thispagestyle{empty}
\newpage
\tableofcontents
% \listoffigures
% \listoftables
\newpage

% begin contents

\section{Introducció}
\label{introduction}
En aquesta pràctica s'ha plantejat la implementació d'una versió simplificada del cas d'ús  \textit{Obtenir Certificació} del Portal Unificat amb els respectius tests. \\

\section{Presa de decissions de disseny}
\label{decissisions}
En aquest apartat s'explicaràn les decissions preses durant la implementació del projecte.

\subsection{Utilització d'enumeracions}
\label{enumeracions}
Per tal d'encapsular la lògica de les diferents constants, com poden ser els diferents mètodes d'autentificació, certificat digital, l'estat en el que es troba l'usuari amb clave permanente, etc. S'han decidit emprar \href{https://docs.oracle.com/javase/tutorial/java/javaOO/enum.html}{enums(enumeracions)}.\\

Aquesta decisió de disseny ha estat presa per evitar el code smell \href{https://refactoring.guru/smells/primitive-obsession}{\textit{Primitive Obsession}} a més d'evitar els \textit{magic numbers} \footnote{Són nombres que els programadors \textit{hardcodegen} al codi i que, a la llarga, fa que no s'entengui el significat d'aquell nombre. Per més informació es pot llegir el \href{https://en.wikipedia.org/wiki/Magic_number_(programming)}{següent article}}. Les diferentes enumeracions han estat afegides dins d'un paquet anomenat \textit{enums}, ja que així es pot veure fàcilment on es troben totes les constants del projecte.

\subsection{Utilització d'interfícies}
\label{interfaces}
S'ha utilitzat el principi SOLID \textit{Interface Segregation Principle} per forçar la implementació dels tests comuns en les classes de testing Cl@ave PIN, Cl@ve Permanente i Certificat Digital. \\

S'han creat tres interfícies per als tests mencionats anteriorment:
\begin{itemize}
  \item \textbf{EnterNifPin:} Conté mètodes que testegen C
  \item \textbf{EnterPin:}
  \item \textbf{UnifiedPlatformTest:}
\end{itemize}


\subsection{Classe Citizen}
\label{citizen}
S'ha requerit la classe Citizen per emmagatzemar les dades necessaries d'un usuari de la plataforma.

\subsection{Classe DNI}
\label{dni}
S'ha requerit la classe DNI per guardar en un mateix objecte el NIF i la data de validació d'aquest.

\subsection{EncryptedData}
\label{encyptedData}


\end{document}
